\documentclass{tufte-handout}
\usepackage{amsmath}
\usepackage{mathtools}
\usepackage{bm}
\usepackage{amsmath}
\usepackage{siunitx}
\sisetup{detect-all}
\usepackage{svg}
\usepackage[utf8x]{inputenc}
%\usepackage[greek,english]{babel} 
\usepackage{textcomp}
\usepackage{textgreek}
%\numberwithin{equation}{section}
\usepackage{physics}%Re,Im not in fraktur
\newcommand{\uvec}[1]{{\bm{\hat{\textnormal{\bfseries $#1$}}}}}
\newcommand{\ux}{\uvec{x}}
\newcommand{\uy}{\uvec{y}}
\newcommand{\uv}{\uvec{v}}
\newcommand{\vv}{\vec{v}}
\newcommand{\ua}{\uvec{a}}
%\DeclarePairedDelimiter\abs{\lvert}{\rvert}% %conflicts with physics
\newcommand{\vac}{V_{ac}}%
\newcommand{\iac}{I_{ac}}%
\newcommand{\wt}{\omega{}t}
\newcommand{\parallelsum}{\mathrel{/\mkern-5mu/}}

\makeatletter
\providecommand\add@text{}
\newcommand\tagaddtext[1]{%
  \gdef\add@text{#1\gdef\add@text{}}}% 
\renewcommand\tagform@[1]{%
  \maketag@@@{\llap{\add@text\quad}(\ignorespaces#1\unskip\@@italiccorr)}%
}
\makeatother

\title{EE1 Simple AC Circuit Analysis Examples}
\author{Professor Timothy Drysdale}
\begin{document}
\maketitle

%\begin{align}
%\end{align}
%\begin{equation}
%\end{equation}

\begin{abstract}
\noindent
We step through the analysis of two example circuits at single frequency
\begin{itemize}
\item series RC
%\item series RL
%\item parallel RC
\item parallel RL
%\item two series voltage sources
%\item series LC
\end{itemize}
\end{abstract}

\section{Series RC}

\begin{marginfigure}
\includesvg{seriesRC}
\caption{Series RC circuit}
\label{fig:seriesRC}
\end{marginfigure}

Let 
\begin{align}
\omega &= 200\pi \\
\vac & = 200e^{jwt}\\
R &= 100\Omega \\
C &= 20\mu F 
\end{align}

\subsection{What else do we know? }
\begin{itemize}
\item \[Z_C = \frac{1}{j\omega C}\]
\item \[ V = IZ \]
\item voltages around a loop must sum to zero, at all times
\item the same current flows in all components of a simple series circuit

\end{itemize}

\subsection{What do we \emph{no}t know?}
\begin{itemize}
\item the current flowing in the supply/resistor/capacitor
\item the voltage across the resistor
\item the voltage across the capacitor
\end{itemize}

One approach is to calculate the total impedance of the circuit at a single frequency, and do all the book work numerically at a single frequency.
\begin{align}
Z_T &= Z_R + Z_C \\ 
& = R + \frac{1}{j\omega C} \\
& = 100 - j80
\end{align}
Then we can calculate the current \marginnote{we multiply by the complex conjugate $(100+j80)$ to make the deonominator real}
\begin{align}
\iac &= \frac{V}{Z_T}\\
&= \frac{200}{100-j80}\\
&= \frac{200(100+j80)}{(100-j80)(100+j80)}\\
& = \frac{20\times10^3+j16\times10^3}{10\times10^3 + 6.4\times10^3} \\
& = 1.22 + j0.98 A
\end{align}
Note that we drop the exponential part of the voltage for convenience. We know it is there, and we'll add it back in later.
What does out result mean? We still need to work out the magnitude $\abs{\iac}$ and phase $\angle{\iac}$. The magnitude is
\begin{align}
\abs{\iac} &= \abs{1.22 + j0.98} \\
&= \sqrt{1.22^2 + 0.98^2} \\
&= 1.56 A.
\end{align}

We can double check our value for $\iac$ by calculating the magnitude of the impedance first
\begin{align}
\abs{\iac} &= \frac{V}{\abs{Z_T}} \\
&= \frac{200}{\sqrt{100^2 + 80^2}} \\
& = \frac{200}{128.1} \\
& = 1.56A
\end{align}
which is the same as before.

The phase is
\begin{align}
\angle{\iac} &= \arctan{\frac{\Im{\iac}}{\Re{\iac}}} \\
&= \arctan{\frac{0.98}{1.22}}  \\
&= \SI{38.8}{\degree}
\end{align}

We can see the results on a phasor diagram of Figure~\ref{fig:seriesRCphasor}.  This is only a qualitative representation - you do not need to draw the phasors to scale in the exam because you will be drawing the phasor diagram \emph{before} you've worked out any values!
\begin{marginfigure}
\includesvg{seriesRCphasor}
\caption{Series RC circuit phasor diagram}
\label{fig:seriesRCphasor}
\end{marginfigure}

We should calculate the voltage across the components. The voltage across the capacitor is 
\begin{align}
V_C &= \iac Z_L \\
&= 1.22 + j0.98 \times -j80 \\
&=  78.4-j97.6  \tagaddtext{[\SI{}{\volt}]}
\end{align}

\begin{align}
V_R &= \iac Z_R \\
&= 1.22 + j0.98 \times 100 \\
&= 122 + j98  \tagaddtext{[\SI{}{\volt}]}
\end{align}

We can check that these sum to give $\vac$
\begin{equation}
78.4-j97.6 +122 + j98  = 200.4+j0.4 \tagaddtext{[\SI{}{\volt}]}
\end{equation}
which is just slightly out due to some rounding error carried through from earlier.

That's a basic solution done and dusted. As we will later see with filters, it can sometimes be convenient to work out the result for the general case, and substitute in numbers as you need them. We'll try that now, just to see what it is like. So let's calculate the current for any frequency and voltage

\begin{align}
\iac & = \frac{V}{Z_T} \\
& = \frac{V}{R + \frac{1}{j\omega C}}\\
& = \frac{jV\omega C}{1 + j\omega RC} 
\end{align}

The magnitude of the current is
\begin{align}
\abs{I} & = \abs{\frac{jV\omega C}{1 + j\omega RC}} \\
&= \frac{\abs{jV\omega C}}{\abs{1 + j\omega RC}} \\
& = \frac{V\omega C}{\sqrt{1 + (\omega R C)^2} }
\end{align}

To find the phase of the current, we need to first make the deonominator real so we can separate out the real and imaginary parts. We multiply by the complex conjugate
\begin{align}
\iac &= \frac{jV\omega C}{1 + j\omega RC}\frac{1-j\omega RC}{1-j\omega RC} \\
& = \frac{V\omega C(1j - \omega RC)}{1+(\omega RC)^2}\\
\end{align}
Now the phase can be found as 
\begin{align}
\angle{\iac} &= \arctan{\frac{\Im{\iac}}{\Re{\iac}}} \\
&= \arctan{\frac{1}{-\omega RC}}
\end{align}

We can check these equations by substituting in the parameters and checking we get the same answers as before. The magnitude of the current is
\begin{align}
\abs{I} & = \frac{V\omega C}{\sqrt{1 + (\omega R C)^2} }\\
&= \frac{200.200\pi.20\times10^{-6}}{\sqrt{1 + (200\pi.100.20\times10^{-6})^2}}\\
&= 1.56 \tagaddtext{[\SI{}{\ampere}]}
\end{align}

The phase of the current is 
\begin{align}
\angle{\iac} &= \arctan{\frac{1}{-\omega RC}} \\
& = \arctan{\frac{1}{-200\pi.100.20\times10^{-6}}}\\
& = \SI{38.5}{\degree}
\end{align}
which is ever so slightly ($\SI{0.3}{\degree}$) different due to the rounding error in our earlier numerical calculations.

\section{Parallel RL}
\begin{marginfigure}
\includesvg{parallelRL}
\caption{Parallel RL circuit}
\label{fig:parallelRL}
\end{marginfigure}

Let 
\begin{align}
\omega &= 2\pi\times10^5 \\
\vac & = 250e^{jwt}\\
R &= 1000\Omega \\
L &= 1mH 
\end{align}

\subsection{What else do we know? }
\begin{itemize}
\item \[Z_L = j\omega L\]
\item \[X_L = \omega L\]
\item \[ V = IZ \]
\item currents into a node equal currents out of a node, at all times
\item voltages across parallel branches are equal, at all times

\end{itemize}

\subsection{What do we \emph{no}t know?}
\begin{itemize}
\item the current flowing in the resistor
\item the current flowing in the inductor
\item the current flowing in the supply
\end{itemize}

\begin{marginfigure}
\includesvg{parallelRLphasor}
\caption{Parallel RL circuit phasor diagram}
\label{fig:parallelRLphasor}
\end{marginfigure}

We know that the voltage must be the same across each branch, so we can calculate the magnitude of the component currents separately and then add vectorially to get the supply current, as shown in Figure~\ref{fig:parallelRLphasor}. The magnitude of the resistor current is
\begin{align}
\abs{I_R} &= \frac{V}{R}\\
&=\frac{250}{1000}\\
&=0.25 \tagaddtext{[\SI{}{\ampere}]}
\end{align}

The magnitude of the inductor current is
\begin{align}
\abs{I_L} &= \frac{V}{X_L}\\
&=\frac{250}{2\pi\times10^5.1\times10^{-3}}\\
&=0.40 \tagaddtext{[\SI{}{\ampere}]}
\end{align}

The supply current is 
\begin{align}
\iac &= I_R + I_L \\
 &= \abs{I_R} - j\abs{I_L} \\
 & = 0.25 - j0.40 \tagaddtext{[\SI{}{\ampere}]}
\end{align}
where we use a $-j$ because CIVIL tells us the voltage must lead the current in the inductor.

Or, taking the same approach as last time, and calculating the total impedance we find
\begin{align}
Z_T &= Z_R\parallelsum Z_L \\ 
& = \frac{j\omega RL}{R + j\omega L} \\
& = 283 +450j\tagaddtext{[\SI{}{\ohm}]}
\end{align}
giving the current as 
\begin{align}
\iac &= \frac{V}{Z_T}\\
&= \frac{250}{283+j450}\\
&=  0.25 - j0.40 \tagaddtext{[\SI{}{\ampere}]}
\end{align}
which agrees with our earlier result.

The magnitude of the supply current is 
\begin{align}
\abs{\iac} &= \abs{0.25 -j0.40}\\
&= \sqrt{0.25^2 + 0.40^2}\\
&= 0.47\tagaddtext{[\SI{}{\ampere}]}
\end{align}

The phase of the supply current is 
\begin{align}
\angle{\iac} &= \arctan{\frac{\Im{\iac}}{\Re{\iac}}}\\
&= \arctan{\frac{-0.40}{0.25}}\\
&= \SI{-58}{\degree}
\end{align}

The waveforms are shown in Figure~\ref{fig:osc} as if they were on the oscilloscope screen.

\begin{figure}
\includesvg[width=\textwidth]{parallelRLosc}
\caption{Waveforms for the parallel RL circuit}
\end{figure}


\end{document}
